\documentclass[11pt,a4paper]{article}


\usepackage[utf8]{inputenc}
\usepackage[T1]{fontenc}
\usepackage[french]{babel}
\usepackage{amsmath,amsthm,amssymb}
\usepackage{algorithm}
\usepackage[noend]{algpseudocode}
\usepackage{graphicx}
\usepackage{listings}
\usepackage{hyperref}
\usepackage{xcolor} 
\usepackage[most]{tcolorbox}
\usepackage{float} 

\title{Pseudo-code}
\author{J. KOZIK}
\date{\today}

\algrenewcommand\algorithmicif{\textbf{Si}}
\algrenewcommand\algorithmicthen{\textbf{alors}}
\algrenewcommand\algorithmicelse{\textbf{Sinon}}
\algrenewcommand\algorithmicend{\textbf{Fin}}
\algrenewcommand\algorithmicdo{\textbf{Faire}}
\algrenewcommand\algorithmicwhile{\textbf{Tant que}}
\algrenewcommand\algorithmicfor{\textbf{Pour}}
\algrenewcommand\algorithmicreturn{\textbf{Retourner}}
\algrenewcommand\algorithmicfunction{\textbf{Fonction}}
\algrenewcommand\algorithmicprocedure{\textbf{Procédure}}
\algrenewcommand\algorithmicrepeat{\textbf{Répéter}}
\algrenewcommand\algorithmicuntil{\textbf{Jusqu'à}}

\definecolor{procedurecolor}{rgb}{0,0,1}
\definecolor{structurecolor}{rgb}{0,0,0.5} 

\begin{document}

\begin{tcolorbox}[title=Structure de la classe Graph,colback=blue!10!white,colframe=blue!95!black]
\begin{algorithm}[H]
\begin{algorithmic}[1]
    \State \textbf{Classe} Graph
    \State \textbf{Attributs} :
    \State \hspace{1em} V : liste des sommets
    \State \hspace{1em} E : liste des arêtes (paires de sommets)
    \State \hspace{1em} G\_nx : graphe networkx (pour l'affichage)
    
    	\\
    	\\


    \Function{get\_neighbors}{v}
        \State l $\gets$ ensemble vide
        \For{chaque e dans E}
            \If{v = e[0]} \State Ajouter e[1] à l \EndIf
            \If{v = e[1]} \State Ajouter e[0] à l \EndIf
        \EndFor
        \State \Return l
    \EndFunction
    \\

    \Function{get\_neighbors\_distant}{v, distance}
        \State l $\gets$ \{v\}
        \If{distance $\leq$ 0}
            \State \Return l
        \EndIf
        \State temp $\gets$ get\_neighbors(v)
        \State l $\gets$ l $\cup$ temp
        \For{chaque vertice dans temp}
            \State l $\gets$ l $\cup$ get\_neighbors\_distant(vertice, distance-1)
        \EndFor
        \State \Return l
    \EndFunction
    \\

    \Function{diametre}{}
        \State len\_max $\gets$ 0
        \For{chaque i dans V}
            \For{chaque j dans V}
                \State d $\gets$ longueur du plus court chemin entre i et j
                \If{d $>$ len\_max}
                    \State len\_max $\gets$ d
                \EndIf
            \EndFor
        \EndFor
        \State \Return len\_max
    \EndFunction
    \\

    \Function{borne}{}
        \State \Return min(ceil($\sqrt{|V|}$), diametre())
    \EndFunction

\end{algorithmic}
\end{algorithm}
\end{tcolorbox}

\newpage

\begin{tcolorbox}[title=Structure de la classe Etat,colback=red!10!white,colframe=red!95!black]
\begin{algorithm}[H]
\begin{algorithmic}[1]
    \State \textbf{Classe} Etat
    \State \textbf{Attributs} :
    \State \hspace{1em} B : ensemble des sommets brûlés
    \State \hspace{1em} NB : ensemble des sommets non brûlés
    \State \hspace{1em} G : graphe associé
    \State \hspace{1em} n : numéro de la séquence (étape courante)
    \State \hspace{1em} B\_v : liste des sommets brûlés par chaque balle
    \State \hspace{1em} C : liste des centres des balles
    \State \hspace{1em} id : identifiant unique
    \State \hspace{1em} score : score de l'état

    \\

    \Procedure{init}{NB, B, G, n, B\_v, C}
        \State self.B $\gets$ copie de B
        \State self.NB $\gets$ copie de NB
        \State self.G $\gets$ G
        \State self.n $\gets$ n
        \State self.B\_v $\gets$ B\_v
        \State self.C $\gets$ C
        \State self.id $\gets$ nouvel identifiant unique
        \State score $\gets$ 0
        \For{chaque $i$ dans $[0, |B\_v|)$}
            \State score $\gets$ score $+$ \Call{score\_sommet}{self.C[i], i}
        \EndFor
        \State self.score $\gets$ score $-$ \Call{diametre\_NB}{\,}$^2$
    \EndProcedure

    \\
    \\

    \Function{score\_sommet}{sommet, n}
        \If{n = 0}
            \State \Return 1

        \Else
            \State \Return $\frac{|\text{voisins à distance } n|}{n^2}$
        \EndIf
    \EndFunction

    \\

    \Function{diametre\_NB}{}
        \State len\_max $\gets$ 0
        \For{chaque $i$ dans NB}
            \For{chaque $j$ dans NB}
                \State d $\gets$ longueur du plus court chemin entre $i$ et $j$ dans $G$
                \If{d $>$ len\_max}
                    \State len\_max $\gets$ d
                \EndIf
            \EndFor
        \EndFor
        \State \Return len\_max
    \EndFunction



\end{algorithmic}
\end{algorithm}

\end{tcolorbox}

\begin{tcolorbox}[title=Structure de la classe Burning\_Number,colback=green!10!white,colframe=green!95!black]
\begin{algorithm}[H]
\begin{algorithmic}[1]
    \State \textbf{Classe} Burning\_Number
    \State \textbf{Attributs} :
    \State \hspace{1em} G : le graphe associé
    \State \hspace{1em} Etat : l'état courant

    \\

    \Procedure{init}{G}
        \State self.G $\gets$ G
        \State self.Etat $\gets$ nouvel Etat initial (tous sommets non brûlés)
    \EndProcedure

    \\
    \\

    \Function{etat\_initial}{}
        \State \Return nouvel Etat initial (tous sommets non brûlés)
    \EndFunction

    \\

    \Function{actions}{etat}
        \State \Return ensemble des sommets non brûlés dans etat
    \EndFunction

    \\

    \Function{propagation}{etat}
        \State Créer copies de NB, B, B\_v, C depuis etat
        \State Incrémenter n
        \For{chaque balle $i$ dans B\_v}
            \For{chaque sommet $v$ brûlé par la balle $i$}
                \For{chaque voisin $l$ de $v$}
                    \If{$l$ n'est pas déjà brûlé par la balle $i$}
                        \State Marquer $l$ comme brûlé
                        \State Retirer $l$ de NB
                        \State Ajouter $l$ à B
                        \State Ajouter $l$ à la balle $i$
                    \EndIf
                \EndFor
            \EndFor
        \EndFor
        \State \Return nouvel Etat avec NB, B, B\_v, C, n
    \EndFunction
    
    \\

    \Function{succ}{etat, action}
        \State Créer copies de NB, B, B\_v, C depuis etat
        \State Incrémenter n
        \If{action déjà brûlé}
            \State Erreur
        \EndIf
        \State Ajouter une nouvelle balle centrée sur action
        \State Mettre à jour B, NB, B\_v, C
        \State \Return nouvel Etat
    \EndFunction

    \\

    \Function{goal\_test}{etat}
        \State \Return Vrai si NB est vide, Faux sinon
    \EndFunction

\end{algorithmic}
\end{algorithm}

\end{tcolorbox}

\newpage

\begin{tcolorbox}[colback=green!10!white,colframe=green!95!black]

\begin{algorithm}[H]
\begin{algorithmic}[1]

    \Function{traiter}{}
        \State niveau $\gets$ borne du graphe divisé par 2
        \State etat\_initial $\gets$ etat\_initial()
        \State noeud\_initial $\gets$ Noeud(etat\_initial)
        \State L\_Noeud $\gets$ [noeud\_initial]
        \State L\_a\_traiter $\gets$ [noeud\_initial]
        \While{L\_a\_traiter non vide}
            \State Courant $\gets$ premier de L\_a\_traiter
            \State Générer les enfants de Courant (propagation puis actions)
            \State Trier les enfants par score
            \State Ajouter la moitié supérieure à L\_a\_traiter et L\_Noeud
            \If{un enfant est but}
                \State Vider L\_a\_traiter
            \EndIf
        \EndWhile
        \State \Return L\_Noeud
    \EndFunction

    \\

    \Function{noeuds\_goal}{L\_noeud}
        \State L\_goal\_noeud $\gets$ []
        \For{chaque n dans L\_noeud}
            \If{goal\_test(n.etat)}
                \State Ajouter n à L\_goal\_noeud
            \EndIf
        \EndFor
        \State Trier L\_goal\_noeud par score décroissant
        \State \Return L\_goal\_noeud
    \EndFunction

    \\
    
\end{algorithmic}
\end{algorithm}

\end{tcolorbox}

\begin{tcolorbox}[title=Structure de la classe Noeud,colback=yellow!10!white,colframe=yellow!95!black]
\begin{algorithm}[H]
\begin{algorithmic}[1]
    \State \textbf{Classe} Noeud
    \State \textbf{Attributs} :
    \State \hspace{1em} Etat : l'état associé au noeud
    \State \hspace{1em} parent : pointeur vers le noeud parent
    \State \hspace{1em} action : action menant à cet état
    \State \hspace{1em} cout : coût pour atteindre ce noeud
    \State \hspace{1em} n : numéro du coup
    \State \hspace{1em} depth : profondeur dans l'arbre

    \\

    \Procedure{init}{Etat, parent, action, cout, n}
        \State self.Etat $\gets$ Etat
        \State self.parent $\gets$ parent
        \State self.action $\gets$ action
        \State self.cout $\gets$ cout
        \State self.n $\gets$ n
        \State self.depth $\gets$ 0
        \If{parent existe}
            \State self.depth $\gets$ parent.depth $+$ 1
        \EndIf
    \EndProcedure

    \\
    \\

    \Function{child\_Noeud}{problem, action}
        \State \Return problem.succ(self.Etat, action)
    \EndFunction

    \\

    \Function{expand}{problem, niveau}
        \State L\_Noeud $\gets$ liste vide
        \State Courant $\gets$ self
        \State Etat\_Intermediaire $\gets$ problem.propagation(Courant.Etat)
        \State L\_actions $\gets$ problem.actions(Etat\_Intermediaire)
        \For{chaque $i$ dans L\_actions}
            \State Etat\_enfant $\gets$ problem.succ(Etat\_Intermediaire, i)
            \State Noeud\_enfant $\gets$ Noeud(Etat\_enfant, Courant, i, Courant.cout $+$ 1)
            \State Ajouter Noeud\_enfant à L\_Noeud
        \EndFor
        \State Trier L\_Noeud par score croissant
        \State \Return moitié supérieure de L\_Noeud
    \EndFunction

    \\

\end{algorithmic}
\end{algorithm}
\end{tcolorbox}

\begin{tcolorbox}[title=Programme principal,colback=gray!10!white,colframe=gray!95!black]
\begin{algorithm}[H]
\begin{algorithmic}[1]

    \State G1 $\gets$ nouveau Graph vide
    \State \texttt{"Instances/graphe.txt"} dans G1
    \State B1 $\gets$ nouvel objet Burning\_Number(G1)
    \State L1 $\gets$ B1.traiter()
    \State L2 $\gets$ B1.noeuds\_goal(L1)
    \State taille\_balle\_min $\gets$ taille minimale des listes C dans L2
	\\

    \State Afficher le nombre total de noeuds retenus (taille de L1)
    \State Afficher le(s) noeud(s) de $b(G)$ minimal :
    
    \\
    \For{chaque l dans L2}
        \If{taille de $l.get\_Etat().get\_C() - 1 = $ taille\_balle\_min}
            \State Afficher l'état correspondant
        \EndIf
    \EndFor

    \State Afficher le graphe G1

\end{algorithmic}
\end{algorithm}
\end{tcolorbox}


\end{document}